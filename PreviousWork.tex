%%%%%%%%%%%%%%%%%%%%%%%%%%%
%%%%%%%%PREVIOUS WORK%%%%%%%%%
%%%%%%%%%%%%%%%%%%%%%%%%%%%

In this chapter, we will go deep in the work that has been done previously in our areas; which concretely is not much if we compare it with more general topics like \emph{text mining}. Besides we will introduce some concepts that will be useful in order to understand where some things come from and where we will be moving next. 


\section{Related work}\label{previousResults}

Due to the technical nature of the work, there are until these days only few authors that refers to this topic, mostly in papers, articles or journals; in which they don't provide that much technical information.\\
Anyways, we will mention here the most important details about some related work that have been done that will help us in the development of this work.\\\\
We will review the work of Robert Schumaker \cite{SCH2012}, \cite{SCH2010}, \cite{SCH2010-1}, which is the closest related work we found. He is an Associate Professor of Management Information Systems at Central Connecticut State University. We will dig into each of the previously cited papers.\\ We will start with the paper named: "Sentiment Analysis of Financial News Articles" \cite{SCH2012}. In this paper, the author mentions that the prediction of the stock price has always had certain appeal to researchers, and then it came the one biggest questions in stock markets:  \emph{Does price history matter?}. According to the author, new information is introduced to the market all the time, while a variety of information sources can all move a stock price, e.g., rumours, eavesdropping and scandals. Financial news articles are considered more stable and a more trustworthy source. We will quote some textual words from the author:

\emph{"However, the exact relationship between financial news articles and stock price movement is complex. Even when the information contained in financial news articles can have a visible impact on a security’s price (Gidofalvi 2001 \cite{GG2001}; Lavrenko, Schmill, Lawrie, Ogilvie, Jensen \& Allan 2000a \cite{LSL2001}; Mittermayer 2004 \cite{MM2004}; Wuthrich, Cho, Leung, Permunetilleke, Sankaran, Zhang \& Lam 1998 \cite{WC1998}), sudden price movements can still occur from other sources, such as large unexpected trades \cite{CW1991}."}

In the previously mentioned work: \emph{"Sentiment analysis of financial news articles"} \cite{SCH2012}, it is not explicitly defined how they got the news articles (manually, by some tool, crawler, etc.). This fact in our work will be important, because as we will show in the results part, our idea will be to retrieve as much articles as possible and to assign a polarity value to each of them. \\

Our results will be compared to the previous work perform by different authors in a period of 5 weeks to the companies from the S\&P 500. For example: Schumaker  \begin{math}\approx 2800\end{math} articles, Mittermayer \begin{math}\approx 6600\end{math} articles, Gidofalvi \begin{math}\approx 5500\end{math} articles.

When we will be talking about the sources of the news articles, we must mention that Schumaker's work \cite{SCH2012}, rely on financial documents from reputable Web sources. There are many financial news aggregation sites to provide this service. One of these sites is Comtex, which offers real-time financial news in a subscription format. Another source is PRNewsWire, which offers free real-time and subscription-based services. Yahoo! Finance is a third such source and is a compilation of 45 different news sources including the Associated Press, we could mention as well Financial Times and PRNewsWire among others. This sources provides a variety of perspectives and timely news stories regarding financial markets.

In our case, we will rely on one reputable source that collect news from around 800 different sources. We are talking about \emph{Google News}.
